\documentclass[a4paper,12pt]{article}
\usepackage[utf8]{inputenc}
\usepackage{dialogue}
\usepackage{geometry}
 \geometry{
 a4paper,
 total={185mm,250mm},
 left=15mm,
 top=20mm
 }

 
\title{Femminicidio - un'analisi}
\author{erolm\_a}
\newcommand{\Walter}{\speak{W}}
\newcommand{\Pollazzi}{\speak{P}}

\begin{document}

\maketitle

\pagebreak

\section*{Incipit}

Il dialogo di seguito riportato è un'analisi proposta al lettore sotto forma di dialogo, come vuole la tradizionale maieutica socratica. È stato ritenuto opportuno scegliere dei nomi fittizi per i personaggi partecipanti e citati; ogni riferimento a fatti o eventi realmente accaduti è puramente casuale.

\section*{Prologo}
Presentiamo qui il caso di un uomo, che chiameremo Walter, condannato dalla Corte di Appello di … per omicidio volontario aggravato di una studentessa universitaria, per vario tempo sua coinquilina, che per comodità sarà di seguito chiamata Teresa. Tale uomo ha commesso 40 giorni prima quello che negli ultimi anni è salito alla ribalta con l'impopolare nome di "femminicidio"; si preferisce astenersi da giudizi in merito a questo reato al fine di mantenere l'interpretazione dei fatti libera da qualsivoglia influenza esterna.

L'associazione ... da anni promuove attività in difesa dei diritti umani, in particolare quelli civili ed etico-sociali; l’aumento della frequenza dei casi di femminicidio ha convinto uno psicologo iscritto all’associazione a iniziare una campagna di sensibilizzazione sull’argomento. Per svolgerla, ha ritenuto necessario redigere un reportage basato sull’esperienza dei diretti interessati, e il caso qui descritto è uno dei tanti che egli ha preso in esame. Per rispetto della privacy chiameremo lo psicologo R. Pollazzi.
Il carcere di ... ha concesso all'associazione il rilascio di un'intervista, purché non si dilunghi per più di venti minuti, vista l'elevata domanda di visite dei familiari dei vari carcerati. Il nostro dott. Pollazzi ha pertanto richiesto di rivedere i termini del permesso potendosi così garantire una serie di interviste intervallate da sole 24 ore.

\pagebreak
\section*{Giorno 1}

\begin{dialogue}

\Pollazzi Buongiorno signor Walter; sono il dottor R. Pollazzi. Io e l'associazione ... siamo lieti con Lei per aver consentito alla partecipazione a quest'esperimento. Spero che Lei si senta allo stesso modo... a proposito come si sente?

\Walter Bene, dottor Pollazzi. Beh, mi rendo conto che questa frase Le apparirà come fatta, e del resto lo è considerando che per quasi 6 settimane mi hanno tenuto in questa gattabuia... vivere qui è un inferno, Lei non ne ha idea!

\Pollazzi Non posso certo contraddirla, signore. Credo che le guardie carcerarie Le abbiano spiegato già le regole del "gioco", ma preferirei fare un veloce ripasso: avremo ogni giorno solo venti minuti per parlare; gradirei che in tale scorcio di tempo Lei si prodighi a rispondere senza commettere atti che potrebbero ridurre o togliere completamente questa chance. Non vorrà di certo che la Direzione le tolga tale privilegio, vero?

\Walter No, no di certo! Iniziamo allora, dottor Strozzino, e veda di sbrigarsi ché il tempo scorre.

\Pollazzi Non così di fretta. Fare un'intervista, specie se di carattere conoscitivo e analitico, in condizioni come questa non è il massimo, pertanto gradirei che il suo equilibrio psicologico si mantenga stabile all'interno di questa chiacchierata. Le posso comunque garantire che, nonostante non stia effettivamente esercitando la mia professione, nessun suo dato sensibile sarà reso pubblico. Nessun familiare della vittima verrà mai a conoscenza di queste interviste, glielo prometto, né il suo né il mio nome comparirà su qualsiasi albo. Si sente ora più sereno?

\Walter ... Mah, spero per Lei che stia ai patti. D'altronde, Lei mi dà l'impressione di una persona seria, quindi Le credo.

\Pollazzi Men male, però sono già passati un minuto e 10 secondi per tali convenevoli. Non acceleriamo la clessidra. Iniziamo?

\Walter E sia.

\Pollazzi Premetto che la Direzione mi ha concesso il permesso di leggere la documentazione riguardante i fatti di cronaca che L'hanno portata qui, quindi non si disturbi a parlarmi dei dettagli del fattaccio; invece, gradirei iniziare da un tema più libero, che la metta a suo agio...

\Walter In altre parole Lei vuole tirare fuori il mio subconscio con una tipica analisi dei sogni freudiana? O vuole fare il classico psicologo e pormi domande banali includendo di test di Rorshark e robaccia del genere?

\Pollazzi La prego di non interrompermi. Il tempo è prezioso, quante volte glie l'avrò detto?

\Walter Tre volte includendo questa.

\Pollazzi Non mi faccia arrivare a una quarta, La prego. Come stavo dicendo, pensavo di iniziare il discorso con l'argomento Liceo.

\Walter Ehm, come mai? Mi aspettavo iniziasse dalla famiglia.

\Pollazzi Vede: Lei è stato uno studente universitario. Ancora oggi, in carcere, continua lo studio delle materie: trovo ammirevole tale sforzo, specie considerando la durata della sua condanna che lascerebbe senza speranze qualsiasi laureando preoccupato per gli appelli e per l'apparenza. Riflettendoci, spendiamo più di un terzo del nostro tempo a studiare, e in certi casi anche di più, specie se la media da mantenere è molto alta e il livello di studi elevato, anche al punto di sacrificare il tempo che di norma sarebbe concesso alla famiglia, agli amici e agli svaghi; Lei rappresenta il classico studente modello, a suo malgrado forse, ma ciò rende quest'analisi molto interessante.

\Walter Immagino dove voglia andare a parare. È credenza di molte persone che non hanno mai visto di presenza tali fatti di cronaca ritenere che ciò sia dovuto a una scarsa istruzione. Poi, questo lo sa bene, ci sono quei bel bontemponi che si sentono intelligenti dopo aver letto o essersi fatti suggestionare da Cavalleria Rusticana, che suppongono le cose più disparate, che so, che questi casi sono un fenomeno del Sud poiché lì si praticherebbe ancora oggi il delitto d'onore. O adesso che viviamo il dramma dell'immigrazione è una buona occasione per sparare alla croce rossa spacciando i gli omicidi di donne come islamici, per dirla alla Perdin\footnote{NdA: si è preferito cambiare il nome di un noto politico}.

\Pollazzi Anche se non volevo minimamente insinuare niente di tutto questo, devo dire che mi sorprende non poco il Suo caso. Così come me anche l'opinione pubblica ne è rimasta sconvolta... \direct{ci scambiamo uno sguardo pesante}. Orsù, partiamo dalla domanda di base: da quanto tempo Lei si accanisce verso gli studi?

\Walter Non saprei, credo da quando so leggere e scrivere.

\Pollazzi Anche dalla prima elementare? Ne è sicuro?

\Walter Certamente: all'età di 6 anni mio padre mi comprava i fumetti di Topolino e I Peanuts, e me li faceva leggere; all'inizio trovavo la sua una forzatura ma nel tempo mi appassionai a quei fumetti così tanto che addirittura facevo a gara con il mio fratellone Carlo a leggere i fumetti e scovare in che vignetta veniva detta una battuta.

\Pollazzi Ok, sappiamo perché ama leggere... ma studiare?

\Walter La mia scuola elementare era veramente da cani. I miei compagni credevano che lo sconto del 50\% di un prodotto volesse dire che il costo era 50€. Lei dirà che è normale per degli alunni alle prime armi, ma gli stessi errori si verificavano pure a Maggio in quinta elementare...

\Pollazzi Evidentemente qualcosa non ha funzionato... che ne pensa?

\Walter Le maestre erano inadatte a quel ruolo, questo è poco ma sicuro. A metà di ogni ora pretendevano di alzarsi per prendere il caffè, e nel mentre i miei compagni cominciavano ad azzannarsi tra di loro...

\Pollazzi Azzannarsi? Li annovera tra le bestie?

\Walter Probabilmente definirli bestie è denigratorio nei confronti degli altri animali. Ciò che succedeva là dentro probabilmente sarebbe stato sufficiente per una denuncia ai danni delle maestre, della Preside e dei genitori... ma gli ultimi erano dei poco di buono, alcuni forse anche più pericolosi di me.

\Pollazzi Del tipo? Cosa facevano?

\Walter La mia classe era etnicamente eterogenea: di 25 bambini, ben 10 erano extracomunitari, Bangladesh e Sri Lanka soprattutto. Nella classe c'erano un paio di bulletti italiani, i cui genitori erano peggio che furfanti, che si divertivano a dare loro fastidio. Non so se fosse per una questione di razzismo in quanto se la prendevano con tutti (me incluso), però con loro erano particolarmente pesanti. Una volta ho assistito perfino a una pisciata, sì ha sentito bene, una pisciata in diretta, e con l'obbligo di succhiare da terra l'urina da terra. Mi rendo conto che la cosa sia piuttosto scabrosa da raccontare, ma è obbligo di ogni analista andare ac pellege.

\Pollazzi Ha ragione, ma le vorrei ricordare che i miei studi concernono psicologia, non psicanalisi.

\Walter E allora perché l'associazione che la manda non ha pensato meglio di mandare qui uno psicanalista o, meglio, un criminologo?

\Pollazzi Questioni burocratiche... per farla breve ero l'unico disponibile... Parleremo di questo un'altra volta: il tempo scorre, abbiamo solo 10 minuti.

\Walter Ma guarda un po', usare il tempo per evadere la mia domanda... comunque dai, mi stavo interessando un poco. Spero solo che alla fine di tutto questo non se ne uscirà con la tipica frase "non è colpa sua, ha solo avuto un'infanzia difficile" o tutte quelle invenzioni e scappatelle legali fatte per scagionarmi.

\Pollazzi La condanna è già avvenuta, e non intendo cambiare il verdetto del
giudice. \direct{perdo 10 secondi per sfogliare gli appunti} Lei non ha parlato però di cosa l'abbia spinta allo studio.

\Walter È vero. Credo, ma non ne sono sicuro, che tutto derivi da un viaggio in campeggio avvenuto nel 2007. Facevo il boy-scout e lì conobbi una bella bambina, Clarissa, di cui diventai cotto in meno di 3 secondi. Di Clarissa mi incuriosivano le sue due passioni: la matematica e i gialli (principalmente i piccoli brividi). Probabilmente ho sempre associato inconsciamente la matematica e la lettura a quella bambina; fatto sta che di ritorno a scuola dopo le vacanze studiai con molto più gusto la matematica, e in breve divenni il più bravo in classe nelle tabelline - anche se, come ho detto, non è che avessi granché di concorrenza.

\Pollazzi Che Liceo ha scelto?

\Walter Ovviamente scientifico. Anche se in seguito ho avuto alcuni ripensamenti, ma nel complesso ritengo di avere fatto una buona scelta.

\Pollazzi Come sono stati gli anni delle medie e del Liceo?

\Walter I primi perlopiù medi/mediocri, i secondi contrastanti.

\Pollazzi Tralasciamo al momento le scuole medie. Per quanto riguarda le superiori, come passava le giornate?

\Walter Studiavo, principalmente.

\Pollazzi Quindi suppongo Lei avesse come obiettivo quello di diventare il primo della classe, non è così?

\Walter Tutt'altro. In realtà essere il primo della classe era forse l'ultimo dei miei pensieri, anche se a dire il vero l'esserlo mi conveniva, dato che ricevevo continuamente borse di studio. Mi piacevano la matematica e la letteratura moltissimo - e in un periodo successivo mi sono avvicinato pure alla storia.

\Pollazzi Immagino che le piacessero così tanto che imparava da autodidatta, non è così?

\Walter Vero, infatti amavo chiamarmi "Leopardi 2000".

\Pollazzi Ha mai partecipato a gare come le cosiddette Olimpiadi di Matematica?

\Walter Certo, e anche a quelle di Statistica. Nelle OliMat prendevo punteggi buoni ma non riuscivo mai ad arrivare al Cesenatico, mentre in quelle di Statistica non sono mai stato un campione in quanto per molto tempo l'ho sottovalutata. Nell'ultimo periodo - in prossimità della Maturità - avevo imparato molte cose interessanti che mi hanno appassionato all'applicazione moderna, il Machine Learning.

\Pollazzi Se non sbaglio è proprio questa la branca di studi che Lei svolgeva all'Università: Informatica con fisica, o in inglese Computer Science and Physics. Mi sono documentato e ho notato che la Statistica fa da padrona tra i requisiti base. Mi dica, come studiava a casa?

\Walter Era un appuntamento fisso: alle 3:30 del pomeriggio stavo davanti i libri a finire i compiti triviali dati da quei spocchiosi dei miei insegnanti, e appena avevo tempo accendevo il PC e stavo tutto il tempo davanti a Wikipedia e YouMath. Mio padre volle farmi un regalo al secondo anno dandomi una collana di libri di Matematica del 1989... the good ol' days, come diceva sempre.

\Pollazzi Non ha però un buon ricordo della scuola?

\Walter In verità no. O meglio, non posso risponderLe con un secco sì o no; come ho detto prima, ho ricordi contrastanti.

\Pollazzi Si spieghi meglio.

\Walter Penso che i licei siano perlopiù negativi in quanto molto spesso i professori tendono a fare "perdere tempo" agli studenti invece di impegnarli in attività produttive: sanno perfettamente che lo studio del latino alle superiori è uno spreco di tempo. Mi ricordo ancora di quando mi fecero fare l'analisi grammaticale del capitolo V de I promessi Sposi... che spreco di tempo. Poi ci sono materie che dimostravano evidentemente l'ignoranza e la testardaggine generale della gente, ad esempio l'inglese. Come insegnante di matematica avevamo un’asina che pretendeva di sentire a memoria i concetti, e si scandalizzava appena sentiva dire che qualcuno era andato più avanti perché la riteneva come un atto di insubordinazione nei suoi confronti. Per contro, ci sono stati altri professori, come quello di Fisica e quello di Chimica, che sono stati come paterni nei nostri - nei miei confronti.

\Pollazzi Sono principalmente d'accordo su Lei tranne che su un punto. \direct{guardo l'orologio} Ci resta solo un minuto. Ha scelto questa facoltà in primis? Ha mai avuto ripensamenti?

\Walter Ho avuto modo di affrontare l'informatica solo sporadicamente. Mi ricordo che imparai il C, linguaggio base per chiunque si avvicini agli studi, soltanto poche settimane dopo la maturità. Ma la decisione l'avevo presa conoscendo uno zio che faceva lo sviluppatore, e vedendo che faceva vari calcoli pensavo fosse un mestiere tagliato per me pensai di iscrivermi a ingegneria. Però poi mi pentii e decisi di passare a Informatica e fisica.

\Pollazzi Lei è un personaggio abbastanza interessante: sono sicuro che avremo tanto di cui parlare. Comunque il tempo è finito. Tornerò tra una settimana. Nel frattempo, ripensi a quanto detto oggi.

\Walter "Vedi oggimai se tu mi puoi far lieto/revelando a la mia buona Costanza/come m’ hai visto, e anco esto divieto; ché qui per quei di là molto s’avanza". Riconosce queste parole?

\Pollazzi Dal contesto deduco si tratti di Manfredi, nel… 3° canto del Purgatorio?

\Walter Esatto. Per farla breve, porti i miei saluti al di fuori, ammesso che qualcuno li accetti mai.

\Pollazzi D'accordo, La ringrazio. Buona giornata e alla prossima settimana.

\end{dialogue}

\pagebreak
\section*{Giorno 2}

\begin{dialogue}

\Pollazzi Buongiorno Signor Walter.

\Walter Buongiorno a Lei, Signor Strozzino neofreudiano. Spero Lei stia bene.

\Pollazzi Certo. E Lei?

\Walter Bene pure. Sa di essere in ritardo, no?

\Pollazzi Mi aspettava?

\Walter Beh, probabile, anzi sicuramente. Non ho uno straccio di niente da fare qui se non studiare e fissare la finestra.

\Pollazzi La comprendo. Dunque, abbiamo 19 minuti e 30 secondi: direi di iniziare.

\Walter Va bene Freud, vuole continuare dalla scuola?

\Pollazzi In verità no. Rileggendo gli appunti della settimana scorsa mi è parso di ricordare che quando Lei ha iniziato a parlare di Clarissa ha fatto una strana smorfia. È stata perlopiù una microespressione, ma credo fosse indice di "seccatura" o addirittura "spiacevolezza", "disgusto". Se lo ricorda?

\Walter Mah, non ci ho fatto caso. Probabilmente ha ragione. Vuole parlare di Clarissa quindi?

\Pollazzi Sì, almeno per ora.

\Walter ... Quindi?

\Pollazzi Mi dica qualcosa Lei, inizi spontaneamente per favore.

\Walter Clarissa, Clarissa, Clarissa... Mi ricordo per quanto tempo abbia fantasticato sulle lettere che componevano il suo nome: "Quanto sono perfette" - mi dicevo - "in paragone a quelle di G... Walter". E le sue scarpine, oh! Mi sembravano dei piedi alati! Oh, quanto avrei voluto possederla tutta per me! Lontana dalle grinfie delle altre ragazze, lontana da quei beoti dei suoi amici-schiavetti!

\Pollazzi Il suo desiderio doveva essere dunque molto forte. Si è mai espresso in qualche modo con qualcuno a riguardo?

\Walter ... No. In quel campeggio non c'era nessun mio conoscente. Peraltro da bambino ero molto introverso pertanto tenni la mia cotta solo per me.

\Pollazzi Lei ha mai cercato di attaccare bottone con Lei, insomma... riuscire a stabilire un dialogo, una relazione?

\Walter Ho sempre voluto ma quegli idioti la seguivano sempre dovunque andasse, tranne quando entrava nella tenda, che era un territorio off-limits per noi maschietti. Ai tempi ero da poco diventato un lupo, e come tale potevo dormire in una tenda leggermente più vicina alle provviste di cibo... bella mossa per chi aveva fame e voleva farsi uno spuntino di mezzanotte! Comunque, mi ricordo che i due dormivano in una tenda vicina - no anzi, accanto la mia, e di notte quando non riuscivano a dormire fantasticavano su di lei nei modi in cui un bambino innocente può fare.

\Pollazzi Lei pure era insonne?

\Walter Sì.

\Pollazzi Quindi non è mai riuscito a parlarle. Immagino che quindi non poteva certo chiederle di fare amicizia con Lei né altro.

\Walter Esattamente... tranne in un episodio.

\Pollazzi Me lo racconti.

\Walter Non credo che sarebbe molto utile ai fini dell'analisi, non è successo niente che abbia influito sulla nostra relazione.

\Pollazzi È vero, ma a me interessa anche l'influenza su di Lei come individuo. Giudicherò da me se quanto Lei dice sia rilevante oppure no.

\Walter Una volta camminavamo in una foresta. C'era un sentiero da seguire, tuttavia era parecchio ripido. Alcuni miei compagni erano stanchi di dover camminare mano nella mano a passo di lumaca sulla ghiaia e pertanto fecero una sorta di gara di velocità che non finì proprio bene. Il caso volle che dei ragazzi nel fare questo cadessero proprio su di lei, che era dietro di me, e che quindi per evitare di perdere l'equilibrio si avvinghiò fortissimo su di me, più o meno all'altezza della milza. Per mia fortuna, nonostante fossi comunque ignaro del gioco che gli altri stavano svolgendo, riuscii non so come a mantenermi in equilibrio, e solo dopo mi resi conto della sensazione di calore che provavo mentre lei mi teneva abbracciato. Notai che, anche dopo che gli altri si furono risollevati, lei fu in qualche modo restia a lasciarmi.

\Pollazzi Quindi, Lei sospettava di essere in qualche modo desiderato?

\Walter Non saprei. Pensavo di sì, anche se a posteriori direi che è più dovuto a un'ammirazione della mia maturità che ben rispecchiava il mio titolo di Lupacchiotto. Comunque, mi sentivo un po' imbarazzato.

\Pollazzi Come mai?

\Walter Beh, non ero io ad abbracciarla, inoltre non potevo girarmi se non col collo: avevo paura che gli altri a un certo punto cantassero quella filastrocca fatta per schernire i fidanzati. \direct{canticchiando} "Siete fidanzati! Siete fidanzati!"

\Pollazzi Capisco\ldots e quindi, Lei voleva ricambiare ma non ci riusciva.
Saprebbe dire cosa la tratteneva?

\Walter Mi imbarazzavo.

\Pollazzi C'erano altre ragazze nella Sua squadra?

\Walter Sì.

\Pollazzi Parlava mai con loro?

\Walter Sì, anzi a dire il vero parlavo molto con loro, anche se nell'ultimo periodo mi sono messo a parlare principalmente con una sua amica e compagna di tenda.

\Pollazzi Lei non era interessato a quella bambina, vero?

\Walter Effettivamente; avrei sempre voluto avere lei tra le mani ma non mi era mai possibile.

\Pollazzi Prima di allora si era mai preso altre cotte?

\Walter Onestamente mai.

\Pollazzi E dopo invece?

\Walter L'ambiente delle elementari non faceva per me. Le ragazze della classe erano molto spesso o dei maschiacci o delle frignone da incubo.

\Pollazzi E dopo?

\Walter Delle medie non ricordo granché. Ok, c'era una ragazza che mi piaceva, ci piaceva la matematica e pertanto quando potevo le parlavo del più e del meno, scusi il gioco di parole.

\Pollazzi E quella ragazza l'ha mai invitata fuori?

\Walter Ai tempi ero sempre troppo timoroso per poter chiedere qualcosa del genere a chiunque, figuriamoci con lei, la mia Beatrice.

\Pollazzi Beatrice? Credeva di essere Dante?

\Walter Una specie.

\Pollazzi Si spieghi meglio, per piacere.

\Walter Ai tempi avevo degli insegnanti che mettevano al primo posto l'aspetto, inteso come "mostrarsi multietnici, aperti al dialogo etc.", e lei era sempre al centro dell'attenzione. Perché lei e io no? Per molti, lei era inoltre "perfetta": alunna modello, brava in matematica e in italiano (migliore della classe insieme a me, tenendo in considerazione che per lei non era la lingua madre). Insomma, perfetta. Un limite asintotico.

\Pollazzi Alle superiori?

\Walter Non che avessi avuto molta fortuna. Anzi, a dire il vero in seconda un mio compagno, nel tentativo di mostrarsi amico nei miei confronti, cercò di farmi rimettere in contatto con lei.

\Pollazzi La ragazza delle medie?

\Walter Sì.

\Pollazzi A proposito, come si chiamava?

\Walter Fatima.

\Pollazzi Ok... quindi cosa ha comportato ciò?

\Walter Che scoprii la mia "predisposizione".

\Pollazzi Ha incominciato a seguirla su Facebook?

\Walter Sì, tuttavia ero troppo, troppo intimorito perché potessi parlarle, anche solo per un secondo. Tutto ciò che riuscii a ottenere fu una mera amicizia su FB. Non molto, mi rendo conto.

\Pollazzi Quindi, Lei la seguiva solo virtuale?

\Walter Non proprio. In verità, quando avevo l'occasione cercavo di recarmi ai vari festival che si organizzavano in città, come lo street food oppure una mostra di arte moderna in un quartiere oramai degradato della mia città: tutto pur di rincontrarla.

\Pollazzi Secondo Lei, lo faceva perché quella ragazza Le piaceva?

\Walter No. Volevo conoscerla meglio, ma al contempo non le avrei mai parlato dei miei veri sentimenti, più di ammirazione che altro.

\Pollazzi Quindi non c’è motivo di definire questo stalking, giusto?

\Walter No. A un certo punto ho smesso, rendendomi conto che stavo mentendo soltanto a me stesso. Però ci fu un momento in cui rischiai di ricadere.

\Pollazzi E quale fu?

\Walter Noi due frequentavamo due scuole diverse. In queste due scuole vi erano due professori di Lettere amici tra di loro che facevano parte di un'associazione chiamata "Spirito e fratellanza", dove si discutevano argomenti filosofici anche se in forma molto light, ovvero in forma di riflessione più che di mero nozionismo tipico di certi professori di storia della filosofia. Nella prima volta che vi partecipai scoprii che anche lei li frequentava, e pertanto promisi con ben poca ponderazione che sarei venuto la settimana successiva.

\Pollazzi E Lei ci venne?

\Walter Eccome. E la rincontrai.

\Pollazzi Il motivo per cui andava in quell'associazione era principalmente lei?

\Walter Inizialmente sì, però poi sono venuto per interesse personale. E ho sempre mantenuto una certa regolarità di presenza, cosa non difficile dato che praticamente tutti, lei inclusa, venivano dal liceo classico mentre io no.

\Pollazzi Quindi, ha smesso di “inseguirla”, giusto?

\Walter Sì. Ho preferito aspettare quasi cinicamente l'Amore, quello vero.

\Pollazzi \direct{Guardo nervosamente l'orologio} Purtroppo vedo che il tempo è finito, e che siamo anzi in ritardo di 20 secondi. Mi dispiace interromperLa bruscamente, ma devo lasciarLa. Spero continueremo domani.

\Walter Arrivederci allora, signor Strizzacervelli. L'aspetterò senz'altro domani.

\end{dialogue}

\pagebreak

\section*{Giorno 3}
\begin{dialogue}
\Walter Buongiorno dottor Pollazzi!
\Pollazzi Buongiorno a Lei. Vedo che finalmente non mi assegna più nomignoli.
\Walter Vuole per caso che continui a chiamarLa Signor Strozzino neofreudiano?
\Pollazzi Con tutta onestà non mi interessa granché.
\Walter E Lei non me lo dà uno? Sono curioso, che genere di soprannome potrebbe mai assumere un "fem-mi-ni-ci-da" \direct{sillabando molto lentamente}?
\Pollazzi Non ci ho pensato onestamente. Suvvia, il tempo stringe.
\Walter Evasione.
\Pollazzi Come?
\Walter Un bel po' di tempo fa mi è capitato tra le mani un'analisi del pensiero di Freud nel mio libro di Filosofia. Certamente Lei ne saprà molto più di me quindi non le farò la lezioncina, ma ricordo di aver letto un passo che mi ha particolarmente stupito. Riassumeva come la coscienza, o per meglio dire l'Io tende a evadere dalla realtà stessa, dai ricordi particolarmente spiacevoli attuando meccanismi detti, per l'appunto, di fuga. Tali meccanismi sono molteplici e spaziano dai semplici lapsus fino alle amnesie a breve e lungo termine e alla regressione infantile. Mi perdoni se ho usato termini imprecisi, del resto non è la mia branca di studi, e d'altronde non mi sono mai avvicinato più di tanto alla psicanalisi.
\Pollazzi Eppure, Lei mi ha appena parlato di Evasione. Certo, è banale la correlazione tra il tempo che fugge o, come direbbe Lei, fugit irreparabile tempus, però ritengo sia interessante scavare di più.
\Walter Cioè? La mia era solo una digressione.
\Pollazzi Lei crede così, tuttavia sono abbastanza sicuro che vi sia dell'altro che Lei sta nascondendo, ed è proprio sulla fuga. Mi dica, a tal proposito, cosa le fa pensare per prima cosa la parola "evasione".
\Walter Ehm...
\Pollazzi Stia attento. Quando Lei dice "ehm..." e perde tempo sta attuando una forma di autocensura. È nostro compito inibire tale processo, pur quanto possa essere scabroso o intimo quello che Lei mi nasconde.
\Walter E vada. Dunque \direct{si prende del tempo, e gli do un’occhiata per non esitare} attenzione, responsabilità, affetto, vagina.. urla, schiamazzi, coltello. Pietre, molte pietre, molte pietre di prigione. Stalin e tutti gli altri dittatori. 1984.
\Pollazzi Ha finito?
\Walter Sì, al momento non c'è altro.
\Pollazzi Ma Lei mi ha parlato poco fa del suo libro di filosofia e di questa digressione. Se permette, vorrei fare una precisazione. Lei ricorderà senz'altro che i meccanismi di fuga si dividono in tre categorie: atti mancati, sogni e nevrosi. Sicuramente nel suo libro di Filosofia era riportata questa classificazione, ma Lei non me l'ha citata. Se la ricordava?
\Walter L'avevo dimenticata, stamente. Adesso me la ricordo. E ricordo pure il frangente: un'interrogazione a un mio compagno era andata male proprio perché si era dimenticato questa classificazione, e il professore l'ha redarguito un bel po': probabilmente per lui Freud era una sorta di Deus ex Machina.
\Pollazzi Quindi Lei sostiene di aver dimenticato la lezione nonostante questo rimprovero indiretto?
\Walter Sì, esattamente.
\Pollazzi Io invece credo l'opposto, ovvero che Lei abbia dimenticato la lezione apposta.
\Walter Ah... e perché mai?
\Pollazzi Come dicevo, la coscienza cerca di dimenticare fatti particolarmente spiacevoli. Probabilmente se Lei si fosse ricordato della tassonomia, per ogni volta che ci avrebbe pensato avrebbe ricollegato quel ricordo, che da quanto mi sembra di vedere le provoca un certo fastidio. Non è così?
\Walter Beh, non avevo mai visto da questo punto di vista la cosa. Detta così, tutto fila.
\Pollazzi Ma Lei ritiene che quel ricordo fosse particolarmente fastidioso?
\Walter Sì. Anche perché a un certo punto l'insegnate disse "E adesso spero che tutti voi ve ne ricorderete". Ah, l'ironia della situazione.
\Pollazzi Lei andava d'accordo con il professore di Filosofia della sua scuola?
\Walter Mah, un pochino. Non è una cattiva persona.
\Pollazzi Mi perdoni. Volevo dire se aveva una buona relazione con lui ai tempi.
\Walter Ai tempi lo consideravo molto freddo e incapace di stabilire una buona relazione con lo studente, come vorrebbe qualsiasi manuale di pedagogia. A volte avevo dei dubbi in merito a qualche brano che leggevo, e avrei sempre voluto chiedergli qualcosa ma era sempre di fretta e quando lo "disturbavo" mi ringhiava sempre con un "ma non sei stato attento alla lezione!? l'ho già detto più e più volte".
\Pollazzi Capisco. Questo insegnante ce l'ha avuto per tutto il triennio?
\Walter Per la verità no. Era proprio dell'ultimo anno.
\Pollazzi Lei ricorda le lezioni con lui?
\Walter Sì, ma ne ho un ricordo molto sbiadito.
\Pollazzi Lei lo ricorda?
\Walter Sì, ma a malapena. Non ricordo neppure il suo nome, per la verità. Non mi vorrà forse dire che...
\Pollazzi Esattamente, Lei ha rimosso una serie di ricordi dalla sua coscienza,
o almeno così crede. In realtà tali ricordi persistono nel suo inconscio.
\Walter In quanto la psicologia dell’individuo sarebbe come un iceberg dove la coscienza rappresenta soltanto la parte galleggiante mentre il resto della persona sprofonda nel mare. Ricordo bene?
\Pollazzi È la stessa schematizzazione di Freud. È sicuro di non aver letto proprio nulla di lui?
\Walter È frutto di una mia ricerca su Wikipedia, proprio per ovviare al fatto che non capivo spesso le lezioni del professore.
\Pollazzi D'accordo. \direct{impiego 10 secondi a guardare i miei appunti}. Quando le ho chiesto di fare brainstorming sulla parola "evasione", Lei ha parlato di tante sensazioni. Vorrei partire dall'attenzione e dalla responsabilità. In base a quanto detto, tutto sembra combaciare. Riguardando le altre parole nella lista, è possibile trovare un filo conduttore comune. Analizziamo ora un altro filo conduttore, che è la sua vita sessuale al liceo. Lei ha mai avuto fidanzate alle superiori?
\Walter Beh, sì, nel senso di "cagnetta frivola" che va tanto di voga nelle superiori.
\Pollazzi Intende dire, che a scuola la gente "perde tempo" a cercare fidanzati o fidanzate solo per soddisfare un desiderio popolare?
\Walter Sì. Effettivamente è un desiderio dettato da un fabbisogno psicologico di controparte sessuale. Me ne sono sempre reso conto.
\Pollazzi Lei non crede che si possa sviluppare un legame profondo a scuola?
\Walter Esattamente.
\Pollazzi Eppure, Lei ha accettato tale rapporto con una di queste ragazze semplicemente perché le conveniva.
\Walter Sì, e la cosa era mutuale. Per lei ero solo un amico da taggare su Facebook, e per me era lo stesso.
\Pollazzi Dunque, quante ne ha avute?
\Walter Tante quante le mie amiche su Instagram.
\Pollazzi Lei è un bell'uomo, quindi lo posso capire. Ma, non eravate quindi interessati a una relazione seria?
\Walter In verità a me sarebbe sempre piaciuto averne una, però le occasioni non c'erano. L'ambiente di quella scuola era principalmente di figli di papà, non che ci fossero chissà quali ragazze con un bel cervello.
\Pollazzi C'è mai stata qualche eccezione?
\Walter Sì. In effetti al terzo anno conobbi una ragazza che mi piacque molto proprio per il suo non essere un'oca. Era della classe di fronte alla mia, e ci conoscemmo "disgraziatamente", nel senso letterale del termine. Infatti per sbaglio io e lei abbiamo avuto uno scontro frontale culminato in una sfortunata testata che ci aveva buttati per terra. All'inizio non la trovai attraente, però mi destò interesse che teneva sotto l'ascella un libro, il Canzoniere. Anche a me piaceva molto Petrarca, e pertanto il giorno dopo la cercai per parlarle con il pretesto di scusarmi per l'inciampo del giorno prima.
\Pollazzi La relazione era, a Suo dire, affiatata?
\Walter Beh, non saprei dirlo. In verità non ci siamo mai detti tra di noi "fidanzati" e neppure ci siamo scambiati un bacio. Entrambi trattavamo la cosa come una semplice amicizia, e per molto tempo l'ho considerata così. Verso la fine dell'anno mi ero lasciato con la mia terza ex e ho di colpo pensato a lei e fantasticato su quanto dovesse essere bello e fico darle un bacio. Pertanto mi scapicollai sullo smartphone per cercarla su Facebook e Instagram, e feci una scoperta che mi buttò nello sconforto: lei si era appena fidanzata,e faceva sostanzialmente le stesse cose che facevo io con la mia ex. La loro relazione non durò molto, ma quando finì io ne avevo già iniziata una con un'altra di cui mi ero innamorato perso. Per la rabbia dovetti interrompere bruscamente la nostra amicizia.
\Pollazzi Questa seconda persona che l'ha interessata particolarmente come la ricorda?
\Walter Oggigiorno direi che è un'idiota, non molto diversa dalle altre oche viziate che spende tutto il suo budget nei vestiti. Ma ai tempi la ritenevo bellissima.
\Pollazzi Quant'è durata la relazione?
\Walter Mah, poco, un mese e mezzo.
\Pollazzi Mi pare però di vedere che Lei non è mai stato profondamente interessato alla ragazza di prima.
\Walter Perché lo dice?
\Pollazzi Come ha appena detto, aveva un'altra occasione per stringere un legame più forte tra voi due, eppure l'ha scartata. In un secondo momento non l'ha ripresa, giusto?
\Walter Esattamente.
\Pollazzi Se permette, credo di avere capito il legame tra evasione e tutte queste ragazze. Lei ha paura di instaurare relazioni serie, durature. Non sa, o non vuole prendersi le proprie responsabilità, e quando ha l'occasione facile la coglie come se stesse perennemente fuggendo.
\Walter Sì Pollazzi. È una cosa alla quale avevo riflettuto tempo fa.
\Pollazzi Ha continuato tale riflessione?
\Walter Beh non ne ho avuto la possibilità, stavo entrando in discoteca.
\Pollazzi Mi fa piacere che adesso qui in carcere ci rifletterà per bene. Oh, a proposito, il tempo è praticamente finito, devo andare.
\Walter Aspetti! C'è una cosa che le volevo dire.
\Pollazzi Mi dica.
\Walter ... No, fa niente, ne parleremo domani.
\Pollazzi Va bene, allora arrivederci!
\Walter Altrettante.
\end{dialogue}

\pagebreak
\begin{dialogue}
 \Pollazzi  Buongiorno Signor Walter. Vedo che è sempre puntuale. Le piace molto conversare con me?
\Walter  Non mi prenda in giro, Pollazzi. Sono qui perché sono interessato. La conversazione di ieri mi ha fatto molto riflettere come non mai. Per la prima volta da quando sono qui devo dire di ringraziare di essere in carcere a studiare: mai ci sono stati periodi della mia vita che abbia dedicato così intensamente alla riflessione, forzata o meno.
\Pollazzi  Mi fa piacere sentire queste cose. Del resto è anche lo scopo delle nostre interviste: capire cosa l'ha portata al femminicidio e analizzare quei pensieri latenti che in realtà sono comuni alla maggior parte di noi che però a un certo punto subiscono un cosiddetto triggering.
\Walter  Prima di cominciare, volevo farLe una domanda ieri, non so se si ricorda.
\Pollazzi  Sì, certo, mi dica pure.
\Walter  Come ha fatto a indovinare che avevo dimenticato la tassonomia di Freud?
\Pollazzi  Non l'ho affatto indovinato. L'ho intuito da Lei.
\Walter  Da me? E come?
\Pollazzi  Ripeteva in modo piuttosto cadenzato e forzoso la nozione di evasione, e da lì ho intuito molto sul Suo metodo di studio, prevalentemente acerbo e pedantesco, dettaglio che Lei stesso mi aveva confermato il giorno prima e dieci minuti dopo. Verso la fine, mentre parlava degli atti mancati, ho notato che era diventato un po' più incerto, al punto che si è addirittura scusato di essere stato impreciso. Ho capito quindi che aveva bisogno di un ripasso.
\Walter  E sapeva pure che avrei parlato del mio professore?
\Pollazzi  Non lo potevo dire con certezza, ma era un'ipotesi che avevo considerato. La parte centrale del processo di analisi, se lo ricordi sempre, è Lei e nessun altro. Io semplicemente cerco di fare le domande giuste e capire la sua intima "versione dei fatti", o meglio la sua visione della realtà.
\Walter  Questo mi consola un po'. E pensare che me ne sarei dovuto rendere conto da me, che testa bacata che sono.
\Pollazzi  Lei non è affatto una testa bacata. È molto difficile analizzare se stessi, e chi lo fa molto spesso non ottiene i risultati sperati. E ora, per favore, continuiamo con l'intervista. Ci restano 17 minuti circa.
\Walter  D'accordo, dottor Pollazzi.
\Pollazzi  Vorrei continuare il tema di ieri. Ho notato che Lei pone molta attenzione ai social media quando si tratta di scegliere un nuovo partner, non è così?
\Walter  Ovvio, del resto non è una cosa che facciamo tutti? Non è questo, alla fin fine, lo scopo primario di Mark Zuckemberg quando lo ha lanciato nella sua università?
\Pollazzi  Stia calmo, signor Walter. Non sto dicendo che né Lei né i suoi
conoscenti abbiate commesso alcunché di male. Invece... \direct{prendo alcuni documenti, tra cui dei fogli stampati} ...questi sono degli screenshot della sua bacheca Facebook. Cosa nota di interessante?
\Walter  Queste sono foto e post recenti risalenti al mio periodo di
fidanzamento con Teresa. In particolare \direct{punta il dito su un post abbastanza singolare, che recita "Finalmente si vive insieme! \#BASTASOLITUDINE !", e poco più in basso un selfie dei due adagiati su un letto} questo è il periodo in cui abbiamo incominciato a vivere assieme. È stato all'incirca un mese dopo che ci siamo conosciuti.
\Pollazzi  Come vi siete conosciuti?
\Walter  All'appello di Febbraio di quest'anno. Dovevo ancora dare un paio di materie, e quel giorno mi toccava un esame di Ricerca Operativa. Assistetti all'orale di Teresa, che prima di allora non conoscevo affatto. Poi toccò a me, e lei mi stava guardando. Non che abbia dato tanto peso alla cosa, infatti subito dopo l'esame me ne andai, ma mi accorsi che lei mi seguiva. Non so se mi crede, ma era lei, non io. Accortomi della cosa abbiamo a parlare, e ci siamo resi conto di essere molto compatibili "come due gocce d'acqua".
\Pollazzi  \direct{prendo i fogli alla ricerca dei post pubblicati durante febbraio} Ricorda qualcuno dei seguenti post?
\Walter  Sì ovviamente. Tra quei post dovrebbe essercene uno che raffigura noi due che studiamo in biblioteca assieme abbracciati. E dovrebbe esserci pure un pezzo di prosa di Baudelaire, Nell'oceano dei tuoi capelli se non erro.
\Pollazzi  \direct{controllo e verifico la veridicità della sua affermazione} Sì, infatti. A Teresa piaceva la letteratura?
\Walter  Sì, molto, infatti ogni tanto ci sfidavamo a ricordare pezzi di prosa, un po' come si fa con le rap battle.
\Pollazzi  Frequentavate lo stesso corso?
\Walter  No. Io frequentavo Informatica e il mio piano di Studi era appunto di tipo statistico perché improntato sul machine learning, mentre lei era studentessa di cibernetica e robotica.
\Pollazzi  Ricerca Operativa è una materia comune?
\Walter  Non saprei, ma di certo lo è per gli interessi suoi, che tendevano primariamente verso lo studio dell'esperienza e dell'apprendimento.
\Pollazzi  Scusi la domanda, ma cosa sarebbe Ricerca Operativa?
\Walter  Si figuri. Detta in poche parole, consiste nell'applicare modelli matematici "discreti" per risolvere problemi di ottimizzazione o di simulazione o di decisione stocastica. Per farle un esempio, immaginiamo che un ente di energia elettrica abbia bisogno di importare petrolio da più paesi. Ognuno di questi paesi impone dei prezzi diversi per il proprio petrolio, però pone delle limitazioni alla quantità che si può comprare per tema di esaurire subito la riserva e di monopolizzare il mercato. Quanto petrolio mi conviene comprare dai vari paesi, sapendo che il costo di raffinazione varia da paese a paese, al fine di massimizzare il numero di galloni con la spesa massima prefissata?
\Pollazzi  Capisco. Ed è difficile?
\Walter  Al contrario, questo è uno dei più facili, e se i paesi sono due si può risolvere mediante un banale piano cartesiano. Problemi del genere li affrontavo già a 15 anni. Per di più, questo è un problema di programmazione lineare, nel senso che si può descrivere un insieme di funzioni, dette funzioni obiettivo, mediante equazioni di primo grado. Spero di non aver esagerato nella spiegazione.
\Pollazzi  Non si preoccupi, penso di aver capito. Non andavo male in matematica.
\Walter  Buon per Lei.
\Pollazzi  Ora comprendo perché Teresa studiasse la materia. Ma tornando a noi. Ora toccheremo un tasto un po' intimo, quindi la prego di mantenere la calma e di collaborare, mi farebbe un gran piacere.
\Walter  Vuole tirare fuori la mia vita sessuale? Almeno non con le guardie in giro, la prego.
\Pollazzi  Le guardie devono prestare il loro servizio. Ma sono abbastanza
convinto che manterranno il loro professionale silenzio, non è così \direct{voltandomi verso le guardie, e loro con un cenno si dichiarano favorevoli}?
\Walter  E vada, dottor Pollazzi.
\Pollazzi  Dunque, quando è iniziata la sua vita sessuale?
\Walter  Al primo anno di università.
\Pollazzi  Come mai non prima?
\Walter  Non ne ho mai avuto possibilità. Non che non ci abbia provato: c'era una c**ona che però non voleva stare con me che mi volevo s***are da un minuto all'altro, ma non ero ricambiato. Per le altre fidanzate, le relazioni duravano troppo poco e molto spesso lei non ci stava.
\Pollazzi  Come mai?
\Walter  Non so. Una volta una mi ha detto qualcosa del tipo "non è che non mi piaci, solo che non ti trovo abbastanza maturo perché tu possa prenderti una responsabilità" e io le chiedevo quali fossero tali responsabilità a cui lei si riferiva, temendo che si potesse riferire alla paternità e avuto conferma di ciò le ho chiesto perché non il preservativo. Alla fine ho capito che era un'oca che stigmatizzava il sesso. Inoltre, sentivo spesso di alcune mie compagne che avevano avuto rapporti con qualche professore.
\Pollazzi  Il rapporto era consenziente, immagino.
\Walter  Sì, ma lo scopo era ovviamente ottenere favori. Tipico favoritismo quello sessuale.
\Pollazzi  Immagino che praticasse l'autoerotismo, no?
\Walter  Sì. Mi masturbavo praticamente ogni giorno, a volte in periodi di noia estrema anche due o tre al giorno. Guardavo porno in streaming e ogni tanto mi s****o sul cellulare quando mettevo la foto di una tipa che volevo farmi.
\Pollazzi  Cosa ne pensava e cosa ne pensa ora della masturbazione?
\Walter  Ai tempi la vedevo come un tabù che dovevo nascondere ai miei genitori. Da quando ho vissuto con Teresa ho considerato quello un "periodo di repressione". Con un partner soddisfacente che valore ha la masturbazione?
\Pollazzi  Quindi quale fu la sua prima esperienza sessuale? Con chi?
\Walter  Fu al primo anno, come ho già detto, con una prostituta. All'inizio non ero favorevole alla cosa, ma i miei amici mi hanno fatto cambiare idea e portato in un bordello.
\Pollazzi  Le piacque quell'esperienza?
\Walter  Onestamente non tantissimo. Vero è che, come si suol dire, la prima volta non si scorda mai, però nel complesso non fui contento della prestazione. E credo che la colpa fu principalmente mia: venni troppo presto.
\Pollazzi  E le volte successive?
\Walter  Ebbi varie storie e qualche rapporto con qualche collega di corso, ma mai mi sentivo soddisfatto, mi sentivo quasi un ninfomane.
\Pollazzi  E con Teresa?
\Walter  Lei era eccezionale a letto: i preliminari li faceva benissimo, al punto che per un suo bacio nell'ascella potevo anche avere un orgasmo. Provavamo tutte le tecniche e posizioni che ci venivano in mente, e niente sembrava fermare la nostra fantasia. Mi ricordo ancora i suoi fellatio prelibati.
\Pollazzi  Pensa di avere migliorato le sue prestazioni sessuali grazie a lei?
\Walter  Sì, molto.
\Pollazzi  E fuori dall'aspetto puramente erotico, voi due vi amavate, giusto?
\Walter  Sì \direct{contrae il viso per un istante}
\Pollazzi  Si sente bene? C'è qualcosa che la mette a disagio?
\Walter  Pensavo agli ultimi momenti, in cui lei mi gridava "Perché, perché mi uccidi? Ti ho sempre amata fino all'ultimo".
\Pollazzi  Me ne vuole parlare?
\Walter  Non credo, non c'è tempo, vede?
\Pollazzi  Sì. Allora le chiederò qualcos'altro. Ha mai avuto storie parallele a quelle con Teresa?
\Walter  Che domande mi fa?!
\Pollazzi  Si calmi signore. Se si urta così devo sospettare la verità delle sue parole. Mi dica per favore la verità.
\Walter  E sia! ... \direct{passano 15 secondi di silenzio estremo} Io l'ho tradita. E lei mi aveva scoperto.
\Pollazzi  L'ha uccisa per mantenere il segreto?
\Walter  Credo di sì.
\Pollazzi  Mi scusi se glielo dico, ma Lei non me la racconta giusta. Ne parleremo domani.
\Walter  Non ha motivi per non credermi \direct{mi sorride beffardamente}. Comunque, a domani Pollazzi.
\Pollazzi  A domani Walter.
\end{dialogue}

\pagebreak

\section*{Giorno 4}
\begin{dialogue}
\Pollazzi  Buongiorno Signor Walter. Vedo che lei è sempre puntuale. Le piace molto conversare con me?
\Walter  Non mi prenda in giro, Pollazzi. Sono qui perché sono interessato. La conversazione di ieri mi ha fatto molto riflettere come non mai. Per la prima volta da quando sono qui devo dire di ringraziare di essere in carcere a studiare: mai ci sono stati periodi della mia vita che abbia dedicato così intensamente alla riflessione, forzata o meno.
\Pollazzi  Mi fa piacere sentire queste cose. Del resto è anche lo scopo delle nostre interviste: capire cosa l'ha portata al femminicidio e analizzare quei pensieri latenti che in realtà sono comuni alla maggior parte di noi che però a un certo punto subiscono un cosiddetto triggering.
\Walter  Prima di cominciare, volevo farLe una domanda ieri, non so se si ricorda.
\Pollazzi  Sì, certo, mi dica pure.
\Walter  Come ha fatto a indovinare che avevo dimenticato la tassonomia di Freud?
\Pollazzi  Non l'ho affatto indovinato. L'ho intuito da Lei.
\Walter  Da me? E come?
\Pollazzi  Ripeteva in modo piuttosto cadenzato e forzoso la nozione di evasione, e da lì ho intuito molto sul Suo metodo di studio, prevalentemente acerbo e pedantesco, dettaglio che Lei stesso mi aveva confermato il giorno prima e dieci minuti dopo. Verso la fine, mentre parlava degli atti mancati, ho notato che era diventato un po' più incerto, al punto che si è addirittura scusato di essere stato impreciso. Ho capito quindi che aveva bisogno di un ripasso.
\Walter  E sapeva pure che avrei parlato del mio professore?
\Pollazzi  Non lo potevo dire con certezza, ma era un'ipotesi che avevo considerato. La parte centrale del processo di analisi, se lo ricordi sempre, è Lei e nessun altro. Io semplicemente cerco di fare le domande giuste e capire la sua intima "versione dei fatti", o meglio la sua visione della realtà.
\Walter  Questo mi consola un po'. E pensare che me ne sarei dovuto rendere conto da me, che testa bacata che sono.
\Pollazzi  Lei non è affatto una testa bacata. È molto difficile analizzare se stessi, e chi lo fa molto spesso non ottiene i risultati sperati. E ora, per favore, continuiamo con l'intervista. Ci restano 17 minuti circa.
\Walter  D'accordo, dottor Pollazzi.
\Pollazzi  Vorrei continuare il tema di ieri. Ho notato che lei pone molta attenzione ai social media quando si tratta di scegliere un nuovo partner, non è così?
\Walter  Ovvio, del resto non è una cosa che facciamo tutti? Non è questo, alla fin fine, lo scopo primario di Mark Zuckemberg quando lo ha lanciato nella sua università?
\Pollazzi  Stia calmo, signor Walter. Non sto dicendo che né lei né i suoi conoscenti abbiano commesso alcunché di male. Invece... \direct{prendo alcuni documenti, tra cui dei fogli stampati} ...questi sono degli screenshot della sua bacheca Facebook. Cosa nota di interessante?
\Walter  Queste sono foto e post recenti risalenti al mio periodo di fidanzamento con Teresa. In particolare \direct{punta il dito su un post abbastanza singolare, che recita "Finalmente si vive insieme! \#BASTASOLITUDINE !", e poco più in basso un selfie dei due adagiati su un letto} questo è il periodo in cui abbiamo incominciato a vivere assieme. È stato all'incirca un mese dopo che ci siamo conosciuti.
\Pollazzi  Come vi siete conosciuti?
\Walter  All'appello di Febbraio di quest'anno. Dovevo ancora dare un paio di materie, e quel giorno mi toccava un esame di Ricerca Operativa. Assistetti all'orale di Teresa, che prima di allora non conoscevo affatto. Poi toccò a me, e lei mi stava guardando. Non che abbia dato tanto peso alla cosa, infatti subito dopo l'esame me ne andai, ma mi accorsi che lei mi seguiva. Non so se mi crede, ma era lei, non io. Accortomi della cosa abbiamo a parlare, e ci siamo resi conto di essere molto compatibili "come due gocce d'acqua".
\Pollazzi  \direct{prendo i fogli alla ricerca dei post pubblicati durante febbraio} Ricorda qualcuno dei seguenti post?
\Walter  Sì ovviamente. Tra quei post dovrebbe essercene uno che raffigura noi due che studiamo in biblioteca assieme abbracciati. E dovrebbe esserci pure un pezzo di prosa di Baudelaire, Nell'oceano dei tuoi capelli se non erro.
\Pollazzi  \direct{controllo e verifico la veridicità della sua affermazione} Sì, infatti. A Teresa piaceva la letteratura?
\Walter  Sì, molto, infatti ogni tanto ci sfidavamo a ricordare pezzi di prosa, un po' come si fa con le rap battle.
\Pollazzi  Frequentavate lo stesso corso?
\Walter  No. Io frequentavo Informatica e il mio piano di Studi era appunto di tipo statistico perché improntato sul machine learning, mentre lei era studentessa di cibernetica e robotica.
\Pollazzi  Ricerca Operativa è una materia comune?
\Walter  Non saprei, ma di certo lo è per gli interessi suoi, che tendevano primariamente verso lo studio dell'esperienza e dell'apprendimento.
\Pollazzi  Scusi la domanda, ma cosa sarebbe Ricerca Operativa?
\Walter  Si figuri. Detta in poche parole, consiste nell'applicare modelli matematici "discreti" per risolvere problemi di ottimizzazione o di simulazione o di decisione stocastica. Per farle un esempio, immaginiamo che un ente di energia elettrica abbia bisogno di importare petrolio da più paesi. Ognuno di questi paesi impone dei prezzi diversi per il proprio petrolio, però pone delle limitazioni alla quantità che si può comprare per tema di esaurire subito la riserva e di monopolizzare il mercato. Quanto petrolio mi conviene comprare dai vari paesi, sapendo che il costo di raffinazione varia da paese a paese, al fine di massimizzare il numero di galloni con la spesa massima prefissata?
\Pollazzi  Capisco. Ed è difficile?
\Walter  Al contrario, questo è uno dei più facili, e se i paesi sono due si può risolvere mediante un banale piano cartesiano. Problemi del genere li affrontavo già a 15 anni. Per di più, questo è un problema di programmazione lineare, nel senso che si può descrivere un insieme di funzioni, dette funzioni obiettivo, mediante equazioni di primo grado. Spero di non aver esagerato nella spiegazione.
\Pollazzi  Non si preoccupi, penso di aver capito. Non andavo male in matematica.
\Walter  Buon per lei.
\Pollazzi  Ora comprendo perché Teresa studiasse la materia. Ma tornando a noi. Ora toccheremo un tasto un po' intimo, quindi la prego di mantenere la calma e di collaborare,
mi farebbe un gran piacere.
\Walter  Vuole tirare fuori la mia vita sessuale? Almeno non con le guardie in giro, la prego.
\Pollazzi  Le guardie devono prestare il loro servizio. Ma sono abbastanza convinto che manterranno il loro professionale silenzio, non è così \direct{voltandomi verso le guardie, e loro con un cenno si dichiarano favorevoli}?
\Walter  E vada, dottor Pollazzi.
\Pollazzi  Dunque, quando è iniziata la sua vita sessuale?
\Walter  Al primo anno di università.
\Pollazzi  Come mai non prima?
\Walter  Non ne ho mai avuto possibilità. Non che non ci abbia provato: c'era una c**ona che però non voleva stare con me che mi volevo s***are da un minuto all'altro, ma non ero ricambiato. Per le altre fidanzate, le relazioni duravano troppo poco e molto spesso lei non ci stava.
\Pollazzi  Come mai?
\Walter  Non so. Una volta una mi ha detto qualcosa del tipo "non è che non mi piaci, solo che non ti trovo abbastanza maturo perché tu possa prenderti una responsabilità" e io le chiedevo quali fossero tali responsabilità a cui lei si riferiva, temendo che si potesse riferire alla paternità e avuto conferma di ciò le ho chiesto perché non il preservativo. Alla fine ho capito che era un'oca che stigmatizzava il sesso. Inoltre, sentivo spesso di alcune mie compagne che avevano avuto rapporti con qualche professore.
\Pollazzi  Il rapporto era consenziente, immagino.
\Walter  Sì, ma lo scopo era ovviamente ottenere favori. Tipico favoritismo quello sessuale.
\Pollazzi  Immagino che praticasse l'autoerotismo, no?
\Walter  Sì. Mi masturbavo praticamente ogni giorno, a volte in periodi di noia estrema anche due o tre al giorno. Guardavo porno in streaming e ogni tanto mi s****o sul cellulare quando mettevo la foto di una tipa che volevo farmi.
\Pollazzi  Cosa ne pensava e cosa ne pensa ora della masturbazione?
\Walter  Ai tempi la vedevo come un tabù che dovevo nascondere ai miei genitori. Da quando ho vissuto con Teresa ho considerato quello un "periodo di repressione". Con un partner soddisfacente che valore ha la masturbazione?
\Pollazzi  Quindi quale fu la sua prima esperienza sessuale? Con chi?
\Walter  Fu al primo anno, come ho già detto, con una prostituta. All'inizio non ero favorevole alla cosa, ma i miei amici mi hanno fatto cambiare idea e portato in un bordello.
\Pollazzi  Le piacque quell'esperienza?
\Walter  Onestamente non tantissimo. Vero è che, come si suol dire, la prima volta non si scorda mai, però nel complesso non fui contento della prestazione. E credo che la colpa fu principalmente mia: venni troppo presto.
\Pollazzi  E le volte successive?
\Walter  Ebbi varie storie e qualche rapporto con qualche collega di corso, ma mai mi sentivo soddisfatto, mi sentivo quasi un ninfomane.
\Pollazzi  E con Teresa?
\Walter  Lei era eccezionale a letto: i preliminari li faceva benissimo, al punto che per un suo bacio nell'ascella potevo anche avere un orgasmo. Provavamo tutte le tecniche e posizioni che ci venivano in mente, e niente sembrava fermare la nostra fantasia. Mi ricordo ancora i suoi fellatio prelibati.
\Pollazzi  Pensa di avere migliorato le sue prestazioni sessuali grazie a lei?
\Walter  Sì, molto.
\Pollazzi  E fuori dall'aspetto puramente erotico, voi due vi amavate, giusto?
\Walter  Sì \direct{contrae il viso per un istante}
\Pollazzi  Si sente bene? C'è qualcosa che la mette a disagio?
\Walter  Pensavo agli ultimi momenti, in cui lei mi gridava "Perché, perché mi uccidi? Ti ho sempre amata fino all'ultimo".
\Pollazzi  Me ne vuole parlare?
\Walter  Non credo, non c'è tempo, vede?
\Pollazzi  Sì. Allora le chiederò qualcos'altro. Ha mai avuto storie parallele a quelle con Teresa?
\Walter  Che domande mi fa?!
\Pollazzi  Si calmi signore. Se si urta così devo sospettare la verità delle sue parole. Mi dica per favore la verità.
\Walter  E sia! ... \direct{passano 15 secondi di silenzio estremo} Io l'ho tradita. E lei mi aveva scoperto.
\Pollazzi  L'ha uccisa per mantenere il segreto?
\Walter  Credo di sì.
\Pollazzi  Mi scusi se glielo dico, ma Lei non me la racconta giusta. Ne parleremo domani.
\Walter  Non ha motivi per non credermi \direct{sorride beffardamente}. Comunque, a domani Pollazzi.
\Pollazzi  A domani Walter.
\end{dialogue}

\pagebreak

\section*{Giorno 5}
\begin{dialogue}
 \Pollazzi Buongiorno signor Walter. Sono contento di vederla qui come al solito ineccepibilmente puntuale.
\Walter  Buongiorno anche a Lei. Sa, potrei dire altrettanto di Lei.
\Pollazzi Ne sono grato. Dunque, oggi vorrei riprendere un discorso iniziato precedentemente ma interrotto a metà. Parlo dell'evasione. Le ricordo il contesto: Lei ha ricordato una digressione del suo libro di filosofia in merito alla psicanalisi freudiana, in particolare il concetto di evasione razionale. Le ho chiesto a quali parole pensava quando nominavo la suddetta parola, e lei mi ha dettato le seguenti parole: \direct{passano 8 secondi perché riesca a trovare il foglio giusto tra i miei appunti} attenzione, responsabilità, affetto, vagina.. urla, schiamazzi, coltello. Pietre, molte pietre, molte pietre di prigione. Stalin e tutti gli altri dittatori. 1984. Non abbiamo però più approfondito l'argomento in dettaglio. Vogliamo incominciare?
\Walter  Sissignore.
\Pollazzi Bene allora. \direct{gli faccio vedere una stringa di carta contenente le parole}  Le chiedo di indicarmi la prima parola dalla quale vuole incominciare. \direct{sceglie la parola 'attenzione'}. Signore, credo ancora fermamente che mi stia nascondendo qualcosa. È probabile che stia cercando di celare un suo pensiero scegliendo questa parola apparentemente al di sopra di ogni sospetto al fine di distogliermi dalla verità. Tenga conto che ciò va contro i suoi stessi interessi, giacché sa perfettamente che non potrei in alcun modo diminuirle o alleviarle la pena. Comunque sia, mi attengo alla sua scelta. Qual è la correlazione tra attenzione ed evasione, a suo avviso?
\Walter  Andiamo! non faccia finta di non saperne nulla. È una cosa che avrà sentito molto spesso. Come direbbe un vecchio rimbambito, "è colpa delle diavolerie di oggi se siete sempre così distratti!". Già prima di entrare in carcere avevo già riflettuto in proposito, se quei vecchi fossero soltanto dei rimbambiti o se magari avessero un barlume di ragione.
Q: E come ha mosso la sua riflessione?
\Walter  In verità all'inizio ammetto di essere stato per così dire molto
biased, però poi il lato più analitico di me ha preferito guardare con dati alla
mano la situazione. Così ho deciso di documentarmi facendo, per esempio, una
stima sul numero di ore passate su Facebook, considerando sia interi periodi che
frazioni di secondo, che potrebbero essere pure il mero guardare il telefonino
durante una lezione, una conferenza, in ascensore, a pranzo o anche in bagno,
insomma in qualsiasi momento. È emerso che, se realizzassimo un istogramma del
numero di ore passate online da un campione di 1000 persone, si otterrebbe una
perfetta campana di Gauss. Perfetta, capisce? La moda sarebbe di 6 ore. Se poi
chiedessimo a quelle stesse persone se, a detta loro, passano troppo tempo
davanti gli schermi quando invece potrebbero passarlo più di presenza cercando
di allontanare i mezzi, si potrebbe realizzare un istogramma maledettamente
simile a quello di cui ho parlato poco fa, in cui la mediana è un sostanziale
"ma che, io?", e prossime a essa le risposte "sì, ma non troppo, alla fine esco
con i miei amici e/o amiche" o "sì, passo decisamente troppe ore, ma non riesco
a farne a meno" . E sembra che il coefficiente di correlazione di Pearson
schizzi alle stelle: addirittura oltre 400, che normalizzato sarebbe all'incirca
il 70\%. È un valore piuttosto sconfortante, anche per chi per molto tempo ha difeso a oltranza la giovane sponda. Quando frequentavo l'associazione "Spirito e Fratellanza" per un mese e mezzo l'argomento conduttore è stato quello della rivoluzione tecnologica e dei social media, e tra me e me riflettevo su queste stime (che avevo già svolte per conto mio) e trovavo sempre più sconfortante quella realtà. Eppure, mi rendevo conto che quando usavo Facebook, è come se di colpo me ne dimenticassi, come se fossi diventato un altro. Una volta scoprii, quasi con orrore, che Facebook aveva effetti latenti su di me, o per meglio dire che anche 12 minuti aver posato il telefono,
non mi sentivo più me stesso. Ragionavo diversamente, guardavo il mondo con l'ottica delle sole meme...
\Pollazzi Quindi ha sperimentato l'alienazione.
\Walter  Sì. Grazie a quello ho capito appieno le parole di Marx e di vari altri sociologi quando si riferivano a questo status mentale.
\Pollazzi E mi dica: Lei non ha mai smesso di usare Facebook nonostante fosse venuto a conoscenza della "cruda realtà", non è così?
\Walter  Dottor Pollazzi... Lei cosa mi sa dire sulla negazione nella psicologia?
\Pollazzi Orbene, la negazione in psicologia consiste nel tentare di sviare volutamente l'attenzione della coscienza quando è presa da che il cervello giudica particolarmente preoccupanti spostandola verso cose più frivole al fine di preservare la salute psicologica dell'individuo. Ovviamente questo può arrivare alle estreme conseguenze come le amnesie ma nel suo caso credo che si tratti più di dipendenza che altro.
\Walter  Lei permette? Non nego che una buona dose della colpa sia di un fattore puramente personale come appunto la dipendenza; però è anche vero che ho letteralmente DIMENTICATO \direct{dice gridando} di aver fatto tutto questo lavoro di calcolo e le considerazioni finora qui esposte. Questo meccanismo non è, in fin dei conti, una RIMOZIONE \direct{dice gridando ancora}? NON LO È? MI RISPONDA, POLLAZZI!
\direct{intervengono le guardie, ma il detenuto si calma in fretta}
\Walter  Dottor Pollazzi... perché stiamo parlando di queste cose che, pur quanto possano essere deprimenti, non hanno alcuna correlazione col femminicidio? Cos'è, ha paura? Non vuole affrontare l'argomento perché le brucia?
\Pollazzi Glielo ripeterò. Lo scopo dello studio di quest'analisi non è tanto la dinamica del femminicidio. Per quella esistono purtroppo per noi centinaia di dossier della polizia. No, io voglio sapere cosa l'ha portata a tale atto perché a me interessa primariamente Lei, la sua psicologia. Quello che ha fatto è solo una conseguenza. Lo vuole comprendere o no?
\Walter  ... No.
\Pollazzi Interessante, si sta comportando come un bambino. Incapace di accettare la verità quando imbarazzante, non è vero signor Walter? Non voglio scomodare la regressione infantile. Lei fa così perché c'è qualcosa che ancora mi nasconde, ed è mio compito metterlo alla luce.
\Walter  Lei mi crede seriamente tanto infantile? Non ho mai sentito parole tanto insulse.
\Pollazzi Continui a considerarmi insulso, ma ricordi bene: io sono dall'altra parte di questo vetro divisorio, e Lei dovrà rimanere qui per almeno 15 anni. Che le piaccia o meno, dovrà fare i conti con se stesso, e io le sto offrendo un modo gentile per farlo. Se non è d'accordo con i miei metodi, chiami le guardie e dica loro che sono un'ospite indesiderato e la chiudiamo qui.
\Walter  ... Mi scusi, dottor Pollazzi. Del resto, non posso fare molto più che sfogarmi e rimpiangere per ciò che ho fatto.
\Pollazzi Le do un consiglio. Probabilmente saprà cosa intendo con il training autogeno, spero.
\Walter  Yep.
\Pollazzi Allora per favore faccia ciò che le dico. Chiuda gli occhi, inspiri ed espiri lentamente. Impieghi circa 15 secondi tra una boccata e l'altra. Lo faccia per tutto il tempo che ritiene necessario.
\direct{passa un minuto}
\Walter  Ma così sprechiamo il nostro tempo!
\Pollazzi Al contrario. Non si preoccupi per il tempo. Domani è sempre un altro giorno, ricorda?
\Walter  Sì. Ha ragione. Allora riprendiamo.
\direct{passa un altro minuto}
\Pollazzi Adesso vogliamo continuare?
\Walter  Sì, ovviamente.
\Pollazzi Vorrei parlare delle mie impressioni su quello che abbiamo discusso oggi. Lei non è stupido e ha già capito, o almeno presunto di capire le cause che l'hanno portata a fare il delitto. Adesso, vorrei che Lei mi parlasse dell'atto. Credo che adesso sia giunto il momento di affrontare l'argomento.
\Walter  D'accordo. Da dove vuole che inizi?
\Pollazzi Lei ha detto ieri che tutto è iniziato da un tradimento. È la verità?
\Walter  ... Sì, cioè, non mi sono spiegato bene.
\Pollazzi Si spieghi allora.
\Walter  In realtà tutto è incominciato con una sorta di gara.  La gara era su quanto potessimo diventare popolari su Instagram nell'arco di un mese. L'idea era originata da un'edizione del telegiornale, avevamo sentito - eravamo già insieme - di una gara tra due ragazzi di questo tipo. Uno aveva guadagnato 145 mila seguaci in un mese, l'altro 139 mila. Nel mese successivo avevano decuplicato queste cifre. Sentendo questa notizia, rimasi un po' scioccato, anche se non lo davo a vedere. Come era possibile ciò? Poi, ho ripensato a un vecchio film e alla teoria centrale, dei sei gradi di separazione. Da lì ho capito che la risposta non era, come pensavo, in un procedimento lineare, ma esponenziale. Una catena di Sant'Antonio che poteva avere tali effetti dopo solo, per l'appunto, sei livelli.
\Pollazzi E dunque avete tentato di emulare la cosa?
\Walter  Pensavamo di sì, però ci siamo resi conto con un paio di conti di cui le parlavo poc’anzi che la cosa sarebbe decisamente andata oltre i nostri limiti, pertanto abbiamo deciso di limitare la sfida a ottenere soltanto 1000 amici su Facebook in un mese.
\Pollazzi Per voi erano importanti questi amici?
\Walter  All'inizio no. Però in seguito, guardando Teresa stare tutto il giorno sul telefono, iniziavo a preoccuparmi per Lei. Però non le dicevo nulla, neanche un mero memento che gli appelli sarebbero stati di lì a poco. E poi, mi resi conto, provai fastidio.
\Pollazzi Fastidio per che cosa?
\Walter  La gara era finita da un po' - l'avevamo interrotta dopo soli 3 giorni - ma lei sembrava sempre così tanto attaccata al telefono che iniziai a sospettare qualche relazione nascosta.
\Pollazzi Quindi era geloso.
\Walter  Sì. Ammetto che una volta riuscii, con una scusa, a guardarle gli storici della chat. Rimasi deluso, perché non trovai nulla se non qualche battuta con un centinaio di amiche nuove.
\Pollazzi Deluso? Credevo che, al contrario, Lei dovesse esserne felice. Non aveva motivo di essere geloso.
\Walter  Sì, lo pensavo anch'io. Però, non saprei. Ci speravo. Volevo rimproverarla in qualche modo. Volevo sentirla mia e avere qualche pretesto per rivendicare la nostra relazione come cosa più importante.
\Pollazzi Capisco. E poi, che è successo?
\Walter  I mesi passavano e diventavo sempre più fissato su ciò che lei faceva. All'inizio il mio controllo su di lei si manteneva subdolo, per esempio tentavo sempre di scoprire in anticipo gli appuntamenti che aveva con le amiche - reali - e trovavo tutti i mezzi per sabotarli, per esempio creando in parallelo miei appuntamenti che dovevano essere irrinunciabile, nel senso "che non poteva rifiutare". All'inizio questa frase non aveva nulla a che vedere con la connotazione mafiosa con la quale è diventata tristemente nota, ma in seguito...
\Pollazzi Il controllo si è fatto meno subdolo da un certo momento in poi. Cosa ha causato tale cambio di rotta?
\Walter  Ci fu un episodio un po' imbarazzante con il quale ebbe luogo il nostro primo litigio. In breve ho tentato, in un momento in cui lei mi dava le spalle, di farle una sorpresa e di penetrarla da dietro. Non mi ero accorto che anche stavolta lei stava chattando con delle sue amiche, e mi rimproverò abbastanza aspramente. Al che io risposi con una certa foga che non trovavo affatto corretto che la mia, mia e soltanto MIA fidanzata - ho ripetuto tre volte l'aggettivo - fosse sempre così tanto su Whatsapp, e ho esternato il finto dubbio "chissà con chi mai parlerai tutto il giorno; magari con un altro fidanzatino da sfruttare, e butti me come un vecchio preservativo, no?". Il mio obiettivo era quello di farla tornare sui suoi passi, secondo un mio progetto. È incredibile il modo in cui lei mi ha permesso di metterle la testa sotto i piedi. Sapevo che in realtà lei mi era sempre stata fedele, ma l'ho sfruttata in proposito.
\Pollazzi E quindi come si è comportata?
\Walter  Ha emesso dei gemiti come una cagna bastonata. Poi ha tentato di baciarmi l'orecchio e di afferrarmi per il pacco, e allora non ho resistito più e il mio impulso animalesco si è sfogato.
\Pollazzi Questo litigio si è ripetuto più volte?
\Walter  Sì. Ogni volta facevamo questa specie di farsa, alla quale lei credeva fino in fondo. Non ho mai capito da dove venisse tutta questa ingenuità. Ogni volta le facevo pesare un difetto, anche il più insulso, l'importante è che io avessi sempre ragione, che fossi perfetto ai suoi occhi.
\Pollazzi Secondo la deposizione, sembra che Lei sia venuto alle mani soltanto per tre volte, coincidenti con le ultime due settimane di vita della vittima. Perché ha ritenuto opportuna la violenza fisica?
\Walter  Ultimamente l'Università ci stava lasciando con l'acqua alla gola, pertanto per continuare i nostri ritmi di studio, che di certo non erano facilitati dalle nostre frequenti litigate, avevo pensato di pianificare la giornata. Non che fosse facile: cucinare, lavare i piatti e casa (l'affitto era suo) ecc. richiedevano tempo che di sicuro non possedevamo. Avevo promesso a me stesso che per una settimana avrei cercato di evitare in qualsiasi modo il litigio con lei. Tuttavia, una volta, mentre lavavo i piatti, la coglievo svenuta dopo aver fatto le pulizie. Immediatamente ho rimangiato la mia promessa e sono andato immediatamente in escandescenza. Ho fatto come un pazzo: le dicevo che era una porca lavativa, che non faceva un tubo dalla mattina alla sera se non chattare con le amiche o frequentare l'università tanto per scaldare una sedia. Lei, di colpo, rispose quasi isterica, e mi ricordo ancora di aver detto "ma come osi rivolgerti così a me!?" e di colpo l'ho strattonata. Poi un altro, poi un altro strattone, finché non l'avevo praticamente gettata a terra come uno straccio. A quel punto la prendevo per i capelli e la sollevavo di peso, e le sussurravo all'orecchio "e ora, smidollata pezzo di straccio che non sei altro, va' a lavorare. O se preferisci, vattene e non farti più vedere".
\Pollazzi Dunque, la cosa si è ripetuta una seconda volta, solo che lei ha deciso a quel punto di andarsene, giusto?
\Walter  Sì. Allora iniziai a seguirla. Come prima, all'inizio ho tentato di essere molto stealth, e solo dopo ho ricominciato a tempestarla di messaggi.
\Pollazzi Quanti giorni sono passati dalla separazione all'omicidio?
\Walter  6 giorni.
\Pollazzi Il resto è storia. Oh, purtroppo mi rendo conto che è tardi, non posso farle più domande. Un attimo solo che annoto una domanda per domani. \direct{scrivo la mia domanda}. Ok, La ringrazio.
\Walter  Grazie a Lei. Buona giornata.
\end{dialogue}

\pagebreak

\begin{dialogue}
 \Pollazzi Buongiorno, signor Walter.
\Walter  Buongiorno a Lei, signor Pollazzi.
\Pollazzi Dunque, vorrei cercare di portare a compimento tutto ciò che abbiamo lasciato in sospeso le ultime volte.
\Walter  Prima vorrei sapere qual è la correlazione tra il negazionismo e il mio caso.
\Pollazzi Con ordine. Mi serve solo un elemento finale, e le prometto che oggi stesso le dirò i risultati del nostro colloquio.
\Walter  Uffa, va bene. Avanti, mi dica cosa le manca.
\Pollazzi È la domanda che mi ero annotato ieri. Lei ha posto, finora, la gran parte delle relazioni in una logica di possesso o di dominio. Di questo ne abbiamo già parlato. Ma non le ho chiesto se Lei, per tutto questo tempo, ha covato qualche forma di sessismo.
\Walter  Mi ci faccia pensare… Alle superiori non ritenevo affatto inferiore a loro, almeno non perché donne. Mi sentivo superiore a molti miei compagni, e questo ha di sicuro contribuito a sviluppare a dismisura il mio ego. Ma non sono di certo sessista.
\Pollazzi Quindi secondo lei non ha ucciso la Sua fidanzata con 40 coltellate in un boschetto nei dintorni, non l’ha mascherata e non le ha messo un cartello al collo con scritto “muori pu***na” perché lei era donna, giusto?
\Walter  Esattamente. L’ho uccisa per i motivi sopra riportati. Oh, andiamo! La smetta di prendermi in giro! So che trova tutto questo assurdo.
\Pollazzi In verità lo trovo semplicemente interessante. Con questa argomentazione lei ha appena stroncato la tesi di parecchi gruppi femministi e pro-donna.
\Walter  Lei è forse un femminista?
\Pollazzi Al contrario. Io ritengo che il primo passo per garantire le pari
opportunità sia, paradossalmente, ignorarle. Le faccio un esempio molto semplice: Lei crede che in paesi moderni come la Finlandia abbia mai previsto delle quote rosa?
\Walter  No, non credo.
\Pollazzi Infatti. Molti Paesi del mediterraneo su questo punto sono così retrogradi da aver bisogno di leggi ad-hoc per garantire quello che, se fosse nella mentalità di tutti, costituirebbe una semplice consuetudine tramandata oralmente.
\Walter  Sono d’accordo con lei. Tuttavia, se non sono sessista, cosa sono?
\Pollazzi Egocentrico e smodatamente bisognoso di riflettori.
\Walter  … Si spieghi.
\Pollazzi Le dirò i risultati delle mie riflessioni dal primo giorno ad oggi.
La prima cosa che ho notato in Lei è la sua spiccata intelligenza, fuori dal
comune anche tra molti diplomati del Liceo Scientifico. Da quanto mi ha
raccontato sembra che l’ambiente che Lei frequentava fosse sostanzialmente
asfissiante, come un freno al suo sviluppo culturale e scientifico. Non ha fatto
altro che attaccare i vari professori della sua scuola, anche quelli che,
secondo Lei stesso, alla fine non erano neanche tanto male ma vi si è comunque
opposto. Un’altra caratteristica interessante è il suo modo di relazionarsi al
social network. Voglio dimostrarLe la profonda correlazione tra i fatti della
scuola e quelli dell’omicidio. Abbiamo detto che Lei, a scuola, cercava
perennemente ragazze da abbordare, ed eventualmente con cui avere un amplesso,
non tanto per soddisfare un suo fine, una sua libido completamente personale, ma
per soddisfarne una indotta, che è quella del bell’apparire nei social network.
La sua bacheca è strapiena di foto di Lei insieme ai parenti, ad amici, nei
gattili municipali pronto a prestare servizio per soccorrere i “poveri gatti non
ancora svezzati”; i numerosi selfie appena uscito dalla scuola, dalla discoteca
o anche dal bagno. I post di acceso scontro verso Perdini e le sue posizioni
palesemente razziste che Lei tanto alacremente criticava. Insomma, per Lei era
fondamentale costruire una personalità basata completamente sull’apparenza. Non
so dirle quanto veramente le piacevano queste cose, ma di sicuro il fine
primario doveva esser questo. Consideriamo ora il momento in cui Lei conosce la
sua vittima. Prima di tutto, il suo interesse per lei era decisamente limitato,
al punto che non è stato Lei a instaurare la relazione. In secondo luogo, nel
periodo in cui avvenivano i fatti violenti che Lei descriveva, non vi è
completamente traccia di post “strani” nella sua bacheca. Ciò è probabilmente
dovuto a un meccanismo di censura tale da nascondere al pubblico cose che Lei
ritiene imbarazzanti o troppo privati come un litigio in casa. Del resto, Lei
non ha mai tolto lo status di fidanzato nemmeno per un secondo. Alla luce di
tutto ciò, sembra evidente che Lei non è caratterialmente molto di verso da
Adolf Eichmann o altri burocrati che hanno fatto le atrocità peggiori soltanto
perché prescritte da un ordinamento giuridico complice. Il suo modo di
relazionarsi con le altre persone è infatti primariamente basato sul programmare
le relazioni, e non ammettere la libertà di azione del partner non perché donna
bensì perché altro, estraneo a lei. Usando le parole di Fromm, il suo modo di
concepire la relazione con tutte le sue fidanzate e amiche del Liceo, e con
Teresa, si basa primariamente sulla modalità dell’avere, ovvero vederlo come un
enorme do ut des, un mero scambio di effusioni atto all’ottenere la
soddisfazione della libido. Che è anche ciò che ci insegna Freud, che siamo
animali profondamente egoisti. Lo si può inoltre cogliere da un dettaglio della
sua vita sessuale: lei non amava ricambiare. Parla sempre e solo di come \emph{lei
le faceva i fellatio}, di come lei **La facesse godere**. Non di come, al
contrario, il vostro rapporto fosse pieno. Perché pieno non \emph{lo era
affatto}.
Inoltre, più volte ha parlato di lei come \emph{la mia donna}, perché lei doveva
\emph{rivendicare i suoi diritti} come se fossero stati messi in discussione
oppure \emph{nelle mie grinfie}. Parlava in termini simili pure della bambina nel campeggio, se ricorda bene.
Ora, so cosa sta pensando. E ha anche ragione. In verità il suo caso è molto comune, perché sono tante le persone che basano la loro relazione col prossimo mediante i parametri appena citati. Non è il solo omicida per futili motivi sentimentali e non sarà l’unico, finché persisterà questo modo di vedere le cose. Lei avrà notato che ho cercato per tutto questo tempo di evitare l’uso della parola femminicida. Tuttavia non è affatto per una questione di paura o perbenismo. Del resto, cosa pensa che mi importi di ciò? La verità è che, se permette, vedo il femminicidio come una conseguenza delle relazioni tossiche. Non è una violenza contro la donna come la si potrebbe immaginare negli anni Cinquanta, nonostante l’apparenza possa suggerire questo. È la violenza contro la libertà in sé, e se per caso gli uomini fossero stati androgini sarebbe stata la stessa cosa. Non è un omicidio contro le donne. Lei stesso, consultando le statistiche, sa di certo che il numero di donne uccise in un anno non si discosta di molto al numero di uomini uccisi da mano violenta. Allora, mi potrebbe chiedere, perché le donne? Ritengo che la risposta vada cercata in un retaggio storico. La donna, in particolar modo quella italiana, si è da poco emancipata e non sempre riesce a rivendicare i propri diritti quando vorrebbe o dovrebbe; non dico che non esistano pure uomini fifoni o incapaci di reagire, ma fatto sta che, vuoi per il motivo appena citato vuoi per gli influssi dei mass media e dei social, lo sviluppo personale è rimasto, sia per uomo che per la donna, fallace. E mentre molti uomini hanno preferito rifugiarsi in una spirale di violenza, altrettante donne hanno continuato, irrazionalmente, a seguire il comportamento dettato da millenni di condizionamento, come per l’appunto un riverbero del riflesso incondizionato. Di certo i cartoni animati nei quali il Principe Azzurro salva la Principessa non hanno fatto altro che peggiorare la situazione, ma attenzione! Non sono un’invenzione recente: molti di quei racconti derivano dalle fiabe dei fratelli Grimm, vissuti in un’epoca dove parlare di diritti per le donne era uno scherzo a volte di cattivo gusto. Né tanto meno ritengo che il femminicidio, letteralmente inteso come “omicidio di donne”, sia un fenomeno recente. Quanti stupri sono avvenuti e avvengono durante le guerre? Quante donne sono state arse al rogo perché dichiarate eretiche da un qualche inquisitore? Quante volte sono avvenuti omicidi in casa perché marito e moglie litigavano e all’uomo era legittimamente concessa la frusta e molto altro? Quante volte le donne sono state condannate a morte perché incolpate di tradimento e adulterio? Dobbiamo necessariamente scomodare la memoria di Enrico VIII, Carlo Magno e tanti altri personaggi “illustri”? \direct{passano 12 secondi di pausa}
\Walter  Dottor Pollazzi, ha finito?
\Pollazzi Sì.
\Walter  Prima di tutto, volevo complimentarmi con Lei per l’ottimo lavoro. Veramente ottimo. Mi ha permesso di riconsiderare parecchie cose che finora le ho dichiarato a cui non ho mai dato tanto peso. Sul discorso dell’avere, gradirei aprire una parentesi che completa il suo discorso. Mi ha ricordato un pezzo del brano di Avere o Essere di Fromm. Sostanzialmente riteneva come la nostra società fosse primariamente costruita sui verbi avere ed essere, che non sussistono allo stesso modo in altre culture, specie quelle orientali. Facendo un po’ di ricerche ho scoperto che, ad esempio, nel russo il verbo essere non esiste perché la copula è implicita, o che in generale nei paesi africani non esiste il verbo avere inteso come appropriamento ma un insieme di verbi dal significato più preciso, ma che non va mai o pltre il concetto di possesso, uso temporaneo, dovere morale o debito. Il verbo avere inteso come proprietà – da distinguersi dal possesso e dall’usufrutto - è un’invenzione propria della cultura romana. Evidentemente questo brano, all’apparenza faceva notare come l’apparente coincidenza avesse in verità cause molto più profonde. Del resto, pensavo tra me e me quando lessi uno stralcio più di 8 anni… e probabilmente se facessi ricerche etimologiche su tutto scoprirei che anche le parole più innocenti in realtà nascondono origini insospettabili.
\Walter  Ahem… Dottor Pollazzi!
\Pollazzi Mi dica.
\Walter  A parte che la ringrazio -  sono contento di vedere un uomo che non fa il classico moralista - ci sono un paio di considerazioni da fare sulla sua analisi. La prima è che, purtroppo, Lei assume che io sia un individuo perfettamente razionale e non sia, nei fatti, psicotico. Né io né gli altri killer. Come può esserne veramente sicuro?
\Pollazzi Il fatto stesso che Lei prenda in considerazione di sua spontanea volontà l’idea della pazzia le fa onore, a dire il vero. In realtà il suo caso è quello di una nevrosi sfociata nell’efferato delitto. Tra Lei e un serial killer come Jack lo squartatore vi è moltissima differenza, e penso Lei stesso se ne ravvederà.
\Walter  In altre parole mi sta dicendo che sono sano di mente?
\Pollazzi Fin troppo.
\Walter  La seconda cosa che volevo farle notare è che, nonostante sia d’accordo con la sua tesi, non sono troppo convinto del modo in cui ci è arrivato. A rigore dovremmo dire che ognuno di noi è materialista nel momento in cui dice “faccio la pasta” anziché “preparo la pasta”? Non è più semplice credere che l’uomo è fondamentalmente pigro e non ci tiene a essere grammaticalmente corretto? E dunque, da dove deduce la sua tesi?
\Pollazzi Ma poco fa è stato Lei stesso a ricordare Fromm.
\Walter  Sì, ma non ho detto di crederci.
\Pollazzi Su questo ha ragione. Infatti non mi sono certo basato su questo. È
stato \emph{lei stesso} a ripetere con incessante enfasi gli aggettivi
possessivi. È \emph{Lei, Lei, sempre Lei} che mi ha suggerito questa chiave di
lettura. In realtà quando ho citato la modalità dell’avere in realtà non pensavo
affatto a questo brano, ma a un significato generale più ampio, che è quello che
alla fine ho spiegato nell’analisi. Non l'ho citato mica io.
Questo suo atteggiarsi, tipicamente petulante, è una forma di difesa per non ammettere il proprio torto.
\Walter  Non è vero.
\Pollazzi Sì invece. In tutte le conversazioni tenute non ha fatto altro che osteggiare la sua conoscenza. Vogliamo parlare dell’ultima terzina del 3° canto del Purgatorio? Il metodo del simplesso? \direct{Walter sembra voler controbattere ma rimane interdetto} Ha ancora qualche dubbio?
\Walter  Sì. Non riesco a ben collocare l’utilità del racconto del campeggio.
\Pollazzi In realtà non è uno dei più strategici, ma denota l’inizio dei suoi guai.
\Walter  Cioè?
\Pollazzi Avrebbe dovuto avvicinarsi a quella girl-scout e avere un atteggiamento più attivo e meno avaro. Deve smettere di aspettarsi che la manna scenda dal cielo e incominciare a reagire, se intende fare sul serio. La società fa schifo, tanto è maschilista. Lo sa, anzi lo sappiamo tutti noi della generazione Z, insieme a tutte le altre; se vogliamo diventare veramente aperti verso il gentil sesso dovremmo innanzitutto smetterla di chiamarlo gentil, smetterla di comportarci da cavalieri (e credo Lei lo sappia già) e infine – adesso le insegno io qualcosa – smettere di configurare pericoli simbolici che rimangono tali. Sradichi tutte le sue ossessioni, le sue manie di dominio, le sue fobie di dominazione e di confinamento, la sua trascendenza, come dire, alfieriana. Sviluppi invece la sua intelligenza emotivo-relazionale, ovvero impari a comunicare col prossimo, non deve continuamente dimostrare niente a nessuno.
\Walter  Ne terrò senz’altro conto. La ringrazio, dottor Pollazzi.
\end{dialogue}

\end{document}
